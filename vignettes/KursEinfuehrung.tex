%  Example for the ETH Beamer template
%  Copyright 2014 by
%  Dr. Antonios Garas,
%  Chair of Systems Design, ETH Zurich
%  Weinbergstrasse 56/58 CH-8092 Zurich
%
\documentclass[
%aspectratio=169,
first,
%handout,
%compress,
%Helv,
ETH3, %ETH1,
navigation
]{ETHbeamerclass}

% Options for beamer:
%
% compress: navigation bar becomes smaller
% t       : place contents of frames on top (alternative: b,c)
% handout : handoutversion
% notes   : show notes
% notes=onlyslideswithnotes
%
\setbeamertemplate{note page}{\ \\[.3cm]
\textbf{\color{colorSG}Notes:}\\%[0.1cm]
{\footnotesize %\tiny
\insertnote}}

\setbeameroption{hide notes}
%\setbeameroption{show notes}

\setbeamertemplate{navigation symbols}{} % suppresses all navigation symbols:
% \setbeamertemplate{navigation symbols}[horizontal] % Organizes the navigation symbols horizontally.
% \setbeamertemplate{navigation symbols}[vertical] % Organizes the navigation symbols vertically.
% \setbeamertemplate{navigation symbols}[only frame symbol] % Shows only the navigational symbol for navigating frames.

\usepackage{amsmath, amssymb}
\usepackage{pifont}
\usepackage{multimedia}

%--------------------------------------------------------------------------------------------------------
%--- Some custom definitions...

\definecolor{cbf}{RGB}{199,100,95}%{163,15,0}
\newcommand{\cbf}{\color{cbf}}

\newcommand{\mean}[1]{\left\langle #1 \right\rangle}
\newcommand{\abs}[1]{\left| #1 \right|}

%--------------------------------------------------------------------------------------------------------
%--- Inputs for the title page...

\title{Z\"uchtungslehre I+II}
\newcommand{\shorttitle}{}

\author{Birgit Gredler and Peter von Rohr}
\date{\today}
\institute{Qualitas AG}

\newcommand{\collaborators}{}
\newcommand{\event}{Folien ZL I+II}
\newcommand{\place}{LFW C11}

\newcommand{\Prob}{\mathrm{Pr}}
\newcommand{\sign}{\mathrm{sign}}
\newcommand{\R}{\mathbb{R}}
\newcommand{\N}{\mathbb{N}}
\newcommand{\nbin}[1]{\mathrm{nb}_\mathrm{in}(#1)}
\newcommand{\nbout}[1]{\mathrm{nb}_\mathrm{out}(#1)}
\newcommand{\one}{\mathbb{\bf 1}}

\newcommand{\sref}[1]{SLIDE \ref{#1}}

%--------------------------------------------------------------------------------------------------------
%--- Here begins the presentation...

\usepackage{Sweave}
\begin{document}
\Sconcordance{concordance:KursEinfuehrung.tex:KursEinfuehrung.rnw:%
1 77 1 1 0 265 1}


\frame{
\maketitle
}
\note{}


\section{Einf\"uhrung in die Vorlesung}

\frame{

  \frametitle{Inhalt}
    \tableofcontents[currentsection]
}
\note{}


\frame{

  \frametitle{Who Is Who}
  \begin{block}{Dozierende}
    \begin{itemize}
    \item Birgit Gredler
    \item Peter von Rohr
    \end{itemize}
  \end{block}

  \begin{block}{Studentierende}
      \begin{itemize}
      \item Studiengang
      \item Erfahrungen mit Tierzucht
      \item Motivation f\"ur diese Vorlesung
    \end{itemize}
  \end{block}

}
\note{}


\frame{
  \frametitle{Ziele der Vorlesung}
  \begin{itemize}
  \item Verstehen der Grundlagen
  \item Erkl\"arung von Zusammenh\"angen
  \item Beurteilung von Aussagen (siehe folgende Zitate)
  \end{itemize}
}
\note{}



\frame{
  \frametitle{Zitate}
  \begin{block}{Zitat 1: Leserbrief im Schweizer Bauer}
  ``[...] Ich habe noch niemanden getroffen, der mir diese Zuchtwerte
  erkl\"aren kann. Eine Kuh von mir hat einen Zuchtwert von $-900$
  und gibt immer noch Milch. [...] ''
  \end{block}

  \begin{block}{Zitat 2:}

  \end{block}
}
\note{}


\frame{

  \frametitle{Administrative Angelegenheiten}
  \begin{block}{Sprache}
    \begin{itemize}
    \item Deutsch
    \item Fachbegriffe Englisch
    \end{itemize}
  \end{block}

  \begin{block}{Webseite}
    \begin{itemize}
    \item Link: http://charlotte-ngs.github.io/LivestockBreedingAndGenomics/
    \item Syllabus, Folien und \"Ubungen werden auf der Webseite verf\"ugbar sein
    \end{itemize}
  \end{block}


}
\note{}


\frame{

  \frametitle{Administrative Angelegenheiten II}
  \begin{block}{Keine Vorlesung am 9. Oktober 2015}
  M\"ogliche Alternativen sind
    \begin{itemize}
    \item Zusatztermin
    \item W\"ahrend dreier Wochen eine Lektion kompensieren
    \item Entscheidung vor 9. Oktober 2015
    \end{itemize}
  \end{block}

  \begin{block}{Kreditpunkte}
  Leistungsnachweis: schriftliche Pr\"ufung am 18.12.2015
  \end{block}


}
\note{}


\frame{

  \frametitle{Vorlesungsprogramm}
  Vorlesungsprogramm auf der Webseite unter ``Syllabus'' verf\"ugbar
  \begin{tabular}{|p{0.5cm}|p{1.5cm}|p{6.5cm}|p{1.5cm}|}
\hline
     & Datum   & Thema & Wer \\
\hline
$1$  & 18.09  & Einf\"uhrung in die Vorlesung, R            & BG \\
     &        & Selektionsindex                             & PvR\\
\hline
$2$  & 25.09  & Selektionsindex mehrere Merkmale            & PvR     \\
\hline
$3$  & 02.10  & Verwandtschaftsmatrix und ihre Inverse      & PvR     \\
\hline
$4$  & 09.10  & keine Vorlesung & \\
\hline
$5$  & 16.10  & Korrektur f\"ur fixe Effekte                & PvR     \\
\hline
$6$  & 23.10  & Varianzanalyse                              & PvR     \\
\hline
$7$  & 30.10  & Varianzkomponentensch\"atzung               & PvR     \\
\hline
$8$  & 06.11  & Varianzkomponentensch\"atzung Teil II       & PvR     \\
     &        & BLUP ein Merkmal                            & BG      \\
\hline
$9$  & 13.11  & BLUP mehrere Merkmale,                      & BG      \\
     &        & wirtschaftliche Gewichte      & \\
\hline
\end{tabular}

}
\note{}

\frame{

  \frametitle{Vorlesungsprogramm II}
  \begin{tabular}{|p{0.5cm}|p{1.5cm}|p{6.5cm}|p{1.5cm}|}
\hline
$10$ & 20.11  & Linkage disequilibrium                      & BG      \\
\hline
$11$ & 27.11  & Genomische Selektion                        & BG      \\
\hline
$12$ & 04.12  & Genomische Selektion                        & BG      \\
\hline
$13$ & 11.12  & Genom-weite Assoziationsstudien             & BG      \\
\hline
$14$ & 18.12  & Pr\"ufung                                   & BG, PvR \\
\hline
\end{tabular}

}
\note{}

\frame{
  \frametitle{Ablauf einer Vorlesung}
  \begin{block}{Typ gem\"ass Vorlesungsverzeichnis}
    \begin{itemize}
    \item Typ G, d.h. Vorlesung und \"Ubung
    \end{itemize}
  \end{block}

  \begin{block}{Unser Anliegen}
    \begin{itemize}
    \item M\"oglichst viel Interaktion, denn so lernen wir am meisten
    \item M\"oglichst wenig Konsumation, denn dabei lernen wir nichts
    \end{itemize}
  \end{block}

  \begin{block}{Ablauf}
    \begin{itemize}
    \item Ab kommender Woche
    \item $3 G = 1 U + 2 V$ wobei $U$: 9-10, $V$: 10-12
    \item Pausen
    \end{itemize}
  \end{block}
}

\frame{
  \frametitle{Voraussetzungen f\"ur diese Vorlesung}
  \begin{block}{Streng genommen}
    \begin{itemize}
    \item KEINE
    \item Grundlegende Begriffe und wichtige Konzepte werden erl\"autert
    \end{itemize}
  \end{block}

  \begin{block}{Hilfreich sind ...}
    \begin{itemize}
    \item Grundbegriffe der quantitativen Genetik (Bachelor)
    \item Grundbegriffe der Statistik (Erwartungswert, Varianz, Sch\"atzung)
    \item Elementare Kenntnisse der linearen Algebra (Vektoren, Matrizen)
    \item Erste Erfahrung mit Programmiersprache (Matlab) oder Statistiksoftware (R)
    \end{itemize}
  \end{block}
}
\note{}


\frame{

  \frametitle{\"Ubungen}
%  \begin{block}{\"Ubungen}
    \begin{itemize}
    \item Zu jedem Vorlesungsblock wird es eine \"Ubung geben
    \item \"Ubungen werden w\"ahrend der \"Ubungsstunde bearbeitet
    \item L\"osungsvorschl\"age eine Woche sp\"ater
    \item Stil der \"Ubungen: Bearbeitung einer Fragestellung mit R (oder anderer Programmiersprache)
    \end{itemize}
%  \end{block}

}
\note{}


\frame{

  \frametitle{\"Ubungen II}
  \begin{block}{Weshalb programmieren?}
    \begin{itemize}
    \item Datenmengen verlangen nach Tools zur effizienten Bearbeitung und Analyse
    \item Flexibilit\"at: Excel und co. sind zu wenig flexibel und zu beschr\"ankt (z. Bsp. Anzahl Zeilen und Kolonnen)
    \item Reproduzierbarkeit und Nachvollziehbarkeit
    \end{itemize}
    $\rightarrow$ besseres Verst\"andnis
  \end{block}

  }
\note{}


\section{Einf\"uhrung in R}


\frame{

  \frametitle{Einf\"uhrung in R}


}
\note{}


\section{Einf\"uhrung in die Theorie des Selektionsindexes}

\frame{

  \frametitle{Einf\"uhrung in die Theorie des Selektionsindexes}


  }
\note{}


\end{document}
